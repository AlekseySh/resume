\cvsection{Work experience}

\cvevent{Staff Data Scientist}
{\href{https://www.newyorker.de/}{NewYorker GmbH}}{March 2021 -- 
present}{Berlin, Germany}
I led a team that works on vector representation for the \textbf{images} 
of the company's product catalog.
I was responsible for research and development and also
communicated with other teams and stakeholders to find the best way of 
using our model.
Finally, we came up with exposing it as an internal API. 
\begin{itemize}
\item Our colleagues boosted metrics in their data science projects via 
using our API. 
\item We also released the training pipeline as open-source 
\href{https://github.com/OML-Team/open-metric-learning}{\underline{library}} 
to improve our employer branding. 
\end{itemize}

\divider

\cvevent{Applied Data Scientist}{Neuromation Inc.}{March 2018 -- October 
2020}{}
Working in a start-up allowed me to broaden my horizons and realize 
projects in machine learning consulting, for example:
\begin{itemize}
\item Built a \textbf{computer vision} system which was able to save 
doctors' time by checking if a box with surgical tools was correctly 
completed. The challenge was to solve the problem without retraining the 
model for the new tools (in a few-shot way).
\item Created a \textbf{text} processing tool to help operators extract 
useful information from emails. It required fulfilling the project from 
communicating with a client and setting up the data labelling process to 
containerization and deploying the trained model.
\end{itemize}

\divider

\cvevent{Junior Data Analyst}{3Red Trading, LLC}{June 2016 -- March 
2017}{}
The company analyzed billions of transactions from the 
\href{https://en.wikipedia.org/wiki/Chicago_Mercantile_Exchange}
{Chicago Mercantile Exchange}.
Based on statistical anomalies, I recognized traders who used automatic 
tools, which allowed our team to predict their behavior.
\smallskip
\smallskip

