\cvsection{Side projects}

% OPEN SOURCE
\cvevent{Open-Source}{\href{https://github.com/OML-Team/open-metric-learning}{\underline{OpenMetricLearning}}, author}{2022 -- present}{}
OML is a framework designed for training and validating vector representation models. I place special emphasis on maintaining well-written documentation and ensuring comprehensive test coverage.

\smallskip
\smallskip
\smallskip

\cvevent{}{\href{https://github.com/catalyst-team/catalyst}{\underline{Catalyst}}, active contributor}{2020 -- 2021}{}
I've
\underline{\href{https://medium.com/pytorch/metric-learning-with-catalyst-8c8337dfab1a}{made}}
a metric learning module for this once-popular analogue
of PyTorch Lightning.

\divider

% TEACHING
\cvevent
{\href{https://research.jetbrains.org/groups/plt_lab/seminars}{Teaching}}
{Teaching at \href{https://harbour.space/}{\underline{Harbour University}}}
{January 2024}
{Bangkok, Thailand}
Neural Networks and Computer Vision, \href{https://harbour.space/data-science/courses/neural-networks-and-computer-vision-nikolenko-shabanov-1011}{\underline{link}}.

\smallskip
\smallskip
\smallskip

\cvevent
{}
{Supervising students from \href{https://www.hse.ru/en/}{\underline{HSE}}}
{2023}
{Remote}
Student graduation projects:
\href{https://github.com/nik-fedorov/term_paper_metric_learning}{\underline{1}}, 
\href{https://github.com/nastygorodi/PROJECT-Deep_Metric_Learning/tree/master}{\underline{2}}.

\divider

% MISC

\cvevent{}
{STIR: Siamese Transformer for Image Retrieval Postprocessing,
\href{https://arxiv.org/abs/2304.13393}{\underline{preprint}}}
{}{}

\cvevent{}{Challenge by Uber: \underline{\href{https://github.com/AlekseySh/uber_competition}{2nd}} out of 100+ teams}
{}{}

\cvevent{}{\href{https://www.kaggle.com/aglasis}{\underline{Kaggle}} Competitions Expert}{}{}

\cvevent{}
{\href{https://youtu.be/O8qtBYeOSKE}{\underline{Lecture}} on Re-Id at JetBrains CS Center}
{}{}
