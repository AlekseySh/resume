\cvsection{Side projects}

\cvevent
{Open-Source activity}
{\href{https://github.com/catalyst-team/catalyst}{\underline{Catalyst}}, In top-10 contributors}{}{}
Catalyst is a popular \textbf{open-source} analogue of Keras for PyTorch (3k stars on GitHub, part of PyTorch ecosystem).
Among other things, I have implemented and tested a metric learning module for obtaining the representations of objects.
\smallskip

\href{https://medium.com/pytorch/metric-learning-with-catalyst-8c8337dfab1a}{\underline{Medium post} [en]}.

\smallskip
\smallskip
\smallskip

\cvevent{}{\href{https://github.com/OML-Team/open-metric-learning}{\underline{OpenMetricLearning}}, author}{}{}
This project is a continuation of the metric learning pipeline which I've implemented in Catalyst.
Note, that the project is under the construction now and I am going to release it soon.


\divider

\cvevent{ML competitions}{\underline{\href{https://github.com/AlekseySh/uber_competition}{Uber Movement Challenge}}. (2nd place out of 100+ teams)}
{11 Oct 2019 -- 10 Feb 2020}{}
This was a tabular competition with the following task: \textit{"Predict when and where road incidents will occur next in Cape Town"}. 
Our strong sides were feature engineering for spatial data and a neat validation pipeline.
\smallskip

\cvevent{}{\href{https://www.kaggle.com/aglasis}{\underline{Kaggle competitions expert}}}{}{}

\divider

\cvevent
{\href{https://research.jetbrains.org/groups/plt_lab/seminars}{JetBrains seminars at CS Center}}
{Gave a lecture about Person Re-Id problem}
{}
{}
{\href{https://youtu.be/O8qtBYeOSKE}{\underline{Video} [ru]}.}
