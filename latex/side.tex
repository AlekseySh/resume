\cvsection{Side projects}

\cvevent{Open-Source}{\href{https://github.com/OML-Team/open-metric-learning}{\underline{OpenMetricLearning}}, author}{}{}
OML is a new framework to train and validate models producing high-quality embeddings.
It has a zoo of pretrained models, Config API, and a rich set of examples.
I pay special attention to maintaining well-written documentation and
covering the whole functionality with tests.

\smallskip
\smallskip

\cvevent{}{\href{https://github.com/catalyst-team/catalyst}{\underline{Catalyst}}, in top-10 contributors}{}{}
It's a popular analogue of Keras for PyTorch (3k stars on GitHub, part of PyTorch ecosystem).
Among other things, I've implemented a metric learning module for it,
\href{https://medium.com/pytorch/metric-learning-with-catalyst-8c8337dfab1a}{\underline{post}}.

\divider

\cvevent{Competitions}{\underline{\href{https://github.com/AlekseySh/uber_competition}{Challenge by Uber}}, 2nd  out of 100+ teams}
{11 Oct 2019 -- 10 Feb 2020}{}
This was a tabular competition with the following task: \textit{"Predict when and where road incidents will occur next in Cape Town"}. 
I organised the team and our strong sides were feature engineering for spatial data and a neat validation pipeline.
\smallskip

\cvevent{}{\href{https://www.kaggle.com/aglasis}{\underline{Kaggle competitions expert}}}{}{}
\divider

\cvevent
{\href{https://research.jetbrains.org/groups/plt_lab/seminars}{JetBrains seminars at CS Center}}
{Lecture about Person Re-Id problem}
{}
{}
{\href{https://youtu.be/O8qtBYeOSKE}{\underline{Video} [ru]}.}
