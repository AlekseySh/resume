\cvsection{Side projects}

\cvevent
{\href{https://github.com/catalyst-team/catalyst}{Catalyst}}
{In top-10 contributors by code amount}{}{}
Catalyst is a popular \textbf{open-source} analogue of Keras for PyTorch (2.3k stars on GitHub, part of PyTorch ecosystem).
Among other things, I have implemented and tested a metric learning module for obtaining the representations of objects.
\smallskip

\href{https://medium.com/pytorch/metric-learning-with-catalyst-8c8337dfab1a}{\underline{Medium post} [en]}.

\divider

\cvevent{\href{https://zindi.africa/competitions/uber-movement-sanral-cape-town-challenge}{Uber Movement Challenge}}
{2nd place (out of 100+ teams)}
{11 Oct 2019 -- 10 Feb 2020}{}
This was a tabular competition with the following task: \textit{"Predict when and where road incidents will occur next in Cape Town"}. We achieved good results due to non-standard feature engineering for spatial data and a neat validation pipeline.
\smallskip

\href{https://github.com/AlekseySh/uber_competition}{\underline{GitHub}}.

\divider

\cvevent
{\href{https://research.jetbrains.org/groups/plt_lab/seminars}{JetBrains seminars at Computer Science Center}}
{Gave a lecture about Person Re-Id task}
{}
{}
{\href{https://youtu.be/O8qtBYeOSKE}{\underline{Video} [ru]}.}
